1)The Shoot:
Whether it's a smartphone made for casual users or a DSLR camera made for professionals, you usually get the best results by choosing a camera that matches your filming needs as well as your budget.
You should make a detailed storyboard before choosing a camera and hardware for a shoot. Knowing exactly what kinds of shots you want should help you choose a camera with the right features. When comparing cameras and hardware, you should consider the following factors:
Auto-focus and manual focus: If you need a point-and-film camera, consider using one with auto-focus. If you want greater control over how your film looks, consider manual focus. Many cameras give you the option between auto and manual.
Frame rate: Frame rate describes how many frames a camera can record in a second. It's often expressed "in frames per second" (fps). Higher numbers of fps usually produce clearer, more detailed images. Most digital cameras have frame rates between 24 and 30 fps; although, some are much higher.
HDSLR: Single-lens reflex (SLR) cameras usually cost less than larger, professional models. HDSLR models improve quality by giving you the option to record high-definition video.
Interchangeable lenses: Cameras with interchangeable lenses give you more control over how your film looks. Wide angle lenses, for instance, have small focal lengths that make them useful when filming landscapes. While useful, collecting interchangeable lenses can become an expensive hobby.
Lighting: Even expensive cameras rarely come with adequate lighting. For good results, purchase an external lighting source made from LED, fluorescent, or tungsten bulbs. Three-point lighting systems have become standard for filmmakers. A three-point system includes a fill light, back light, and key light.
Light stands: Stands can make your lights steady so that they produce a more professional appearance. Light clamps and booms can also work well.
Lighting filters: You can improve the quality of your lighting with filters, diffusers, and gels that change the color and direction of light.


Before you can edit digital film, you need to transfer it from your camera to your computer. The transfer method depends somewhat on the type of camera and computer that you have. At first, you may find transferring film difficult, but it gets easier once you have found an option that works well for you.
Transferring Film to an Apple Device:
Connections for Apple computers have evolved quickly. Newer computers, iPads, and other Apple devices have switched from standard USB ports to smaller lightning ports. Since many cameras use USB connectors, Apple users may need a camera connection kit that makes USB cables compatible.
Once you connect the camera to your device, a dialogue box should open that lets you import files to the computer's Camera Roll. Apple computers can use a variety of importing software. Popular importing software for film includes iPhoto, Image Capture, and Aperture.
Transferring Film to a PC:
In most cases, Windows recognizes when you attach a removable disk to the computer. Windows should open a dialogue box that lets you move files from the camera to your folder.
Memory cards: If your camera has a memory card (and your computer has a built-in reader for that type of card), then you can transfer film by inserting the card into the proper port on your computer. If your computer doesn't have the right kind of port, you can usually find a USB port adaptor that'll work.
USB port: Many digital cameras come with USB cords. Plug the cord into your camera and computer to establish a connection that lets you transfer information quickly.
FireWire: Older digital cameras may use FireWire cables, also known as iLink and IEEE 1394 cables. Modern computers usually lack a FireWire port, though. If your camera uses FireWire, then you may need an adaptor.

2)The upload:
There are literally hundreds of video editing programs. Some of them cost thousands of dollars, while others are completely free. When choosing editing software, consider what features you want to use. This should help you focus on an option that meets your needs. Then, you can narrow the list down to software that matches your budget.
If you're just getting started, try free programs that'll acquaint you with how editing software works. Once you gain more experience, explore paid software that offers more powerful features.
When choosing editing software, consider some of these popular options. You can find more options here.
Ezvid: This free video editing software gives you the basic features needed to make a good video. You can use it to splice video, change video speed, and record audio to add to the video. Unlike other free software, Ezvid has a speech synthesis feature that converts text to computer language. It works on most Windows platforms.
Avid Free DV: This free video editing software works on Windows and Mac operating systems. The program lets you edit video and audio using real-time effects. A paid version called Avid Studio is also available.
VideoSpin: This is an extremely basic free video editor with some stability issues. You can upload completed video directly to YouTube, but the software has limited features suitable only for beginners.
Windows Movie Maker: This software's flexibility makes it useful with most video formats. Some editors complain that it has limited functionality. Beginners and intermediate users, however, can use the program to learn how to apply effects, edit for content, and add audio.
iMovie: The iMovie video editing software comes free with many Apple products. It's simple to use, but it still has some advanced features that'll appeal to intermediate filmmakers. With this software, you can easily arrange clips on a timeline, crop frames, and add transitions between scenes. It's very powerful considering it's free.
Wondershare Video Editor: This software offers beginning video editors the tools they need to make high-quality videos. It has an attractive, simple interface that's suitable for all skill levels. If you're new to editing video, this powerful program will get you started and teach you how to use common tools that you'll find in more advanced software. For less than 50, it's hard to find a better editor than Wondershare.
CyberLink PowerDirector: At first glance, CyberLink PowerDirector seems like any video editing software that costs about 50. A closer look shows that it works as a teaching tool as well as an editing program. You get to choose between three modes   (full feature editor, easy editor, and slideshow creator). Each mode focuses on tools that fit your skill level. That means beginning editors don't get confused by advanced processing features, while advanced users don't get bored using the same old tools. The software comes with more than 300 effects and lets you use up to 100 tracks at a time.
Corel VideoStudio: Corel VideoStudio has a simple interface that makes it easy to use. The software typically sells for less than 100, but it comes with powerful tools that match practically any skill level. The software lets you import files from smartphones as well as DSLR cameras. The tools let you add titles, clips, and filters. You can merge clips and even use it to make stop-motion movies. Afterward, you can use the program to share your work online.
Adobe Premiere Elements: This affordable editing software typically sells for less than 100. That small price gives you a full suite of video and audio editing features. Premiere Elements lets you add video effects, transitions, subtitles, and more. You can even use it to make a DVD copy of your movie.
Final Cut Pro: The newest editions of Apple's Final Cut editing software can cost more than 1,000. For that reason, it's most appealing to professional video editors who want extremely powerful features. The software is compatible with most video formats. It can even work with multiple formats at once. Filmmakers who want to use multiple cameras should consider this option, since it makes it relatively easy to keep track of more than one video stream during non-linear editing. It also has color correction, video transitions, and audio filters for a professional product.
Don't assume that the most expensive or most powerful video editing software is best for you. It makes sense to choose editing software that meets you at your own skill level. Starting with professional software that expects you to know a lot about video editing could make you feel overwhelmed and unmotivated. If you start with a simple program, you'll get accustomed to using its tools. Eventually, you may want to learn more so you can use software with more options.

\section{The Export}
Exporting video from your editor makes it possible for computers and websites to read your files properly. Exporting essentially translates the video and audio information from your editing software's unique files to common formats that media players understand.
There are many file formats that you may want to use. It's best to know how you plan to use the video before exporting to a new format. Some media players can only read specific formats. Others may get confused when asked to play unfamiliar formats.
To get the best results, follow these exporting practices for the player you plan to use.
Tip: Always save your original video files before exporting to a new format. Keeping the originals gives you a chance to make further edits to the video. You never know when you might want to put the video in a new format. Saving the original files also protects your hard work from unexpected events like computer crashes.
YouTube and Vimeo both support these types of formats:
MOV, a format native to QuickTime
MPEG-4, a compressed audio-video format used in streaming media, CD recordings, and broadcast TV
AVI (Audio Video Interleave), a format developed by Microsoft that can contain audio and video files
WMV (Windows Media Video), a video compression format created by Microsoft
MPEG-PS, a container format developed by the Moving Pictures Experts Group and primarily used for DVD and HD DVD
FLV, Flash Video format created by Adobe Systems for audio, video, text, and data
WebM, royalty-free audio-video format developed for HTML5
Exporting for YouTube
While YouTube accepts several formats, it's best to use MP4. When exporting a video to YouTube, you'll get the best results by choosing frame rates between 25 and 30. A 16:9 resolution like 640 x 360 also helps make your video clearer. You may need a H.264, MPEG-2, or MPEG-4 codec for the exported video to work properly in YouTube.
The site doesn't have a recommended bit rate, so feel free to use your preference. Since sound quality can affect how people perceive video, choose a 44.1 kHz sample rate.
Exporting for Vimeo
When exporting video for Vimeo, you can choose from a variety of frame rates, including 24, 23.976, 25, 30, and 29.97. Use a 640 x 480 resolution for standard- definition video and a 1280 x 720 resolution for HD video. For the video bit-rate, it's best to select 2,000 Kbps for standard-definition video and 5,000 Kbps for HD video.
You can give the video quality audio by using a 320 Kbps bit-rate and 48 kHz audio sample rate. You'll need the AAC-LC audio codec to export properly for Vimeo.
Exporting for DVD
The format that you use for DVD largely depends on the type of DVD player you plan to use. Nearly all DVD players, whether designed for PAL or NTSC, can read the MPEG-2 format, so it's best to use it. NTSC is used in most North and South American countries. PAL is used in Europe, Australia, and most parts of Asia.
If exporting for NTSC DVD, you'll typically get the best results by using 720 x 480 image resolution, 29.97 fps, and 6,000 Kbps video bit-rate. Audio should use a 224 Kbps bit-rate.
If exporting for PAL DVD, use a 720 x 576 image resolution and a 25 fps PAL is similar to NTSC in that they both use a 6,000 Kbps video bit-rate and a 224 Kbps audio bit-rate.
The results can vary depending on your camera, editing software, and other factors. Start with these best practices before making adjustments that give your video the look that you prefer.
\section{The Upload}
Once you have your video edited and exported, you can share it on a website or social media platform like Instagram or YouTube. Each service has its own protocol for uploading content, so make sure you understand how to use your favorite sites.
Using YouTube Video Manager
If you don't already have a YouTube account, you'll need to sign up here. Since Google owns YouTube, a Gmail, Google Drive, or other type of Google account will also work.
Click the "upload" button on YouTube's homepage.
Select the video you want to submit to YouTube. While the video uploads, you can add metadata such as a description of the video, its title, and tags.
Click the "publish" button to make the video accessible online.
To make the video available to anyone online, make sure you deselect "private" and "unlisted." If you only want certain people to see the video, then make sure to set your account accordingly by following these directions.
Using YouTube Uploader
The free YouTube Uploader can make the uploading process easier, especially if you want to upload several videos at once.
Install Free YouTube Uploader on your computer.
Open the program so you can drag and drop files into the dialogue box.
Click the "video info" button so you can give the file a title and edit properties.
If uploading multiple videos at once, right-click on the dialogue box to set several titles and properties at the same time.
Click the "upload" button, use the dialogue box to sign in to YouTube, and wait for the videos to upload.
Uploading Video to Instagram
While Instagram members can record video directly to their accounts, it's also possible to upload video saved on a device. Many people upload video from their mobile devices, such as iPhones, iPads, and Android smartphones.
Open the Instagram app in your device.
Tap the camera icon to switch to video.
A box will appear at the bottom right of your screen. Tap it to access your device's video album.
Choose the videos you want to upload to Instagram.
Since you can only post up to 15 seconds of video on Instagram, use the video strip to select the portion you wish to upload.
On Apple devices, tap "add" to start the upload. On Android devices, tap the arrow at the screen's top.
Once you upload your video to a website or similar platform, you can share it with the world by sharing the video's link or publishing it on a social media site. Some sites even let contributors earn money when people watch their videos.
It's unlikely that your video will have social or financial success until you learn how to use tools for shooting, transferring, editing, exporting, and uploading your content. However, with a little patience and a lot of practice, your videos will stand a much better chance of garnering praise as well-edited, professional-looking masterpieces
